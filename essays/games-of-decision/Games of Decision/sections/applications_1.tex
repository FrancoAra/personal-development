One of the most important applications of decision systems in human communities is government, in order to model a decision system using our model we just need to define $\alpha$ and $\delta$; in this section we will use our decision systems abstraction to model three well known political systems: autocracy, democracy and representative democracy.

\subsection{Autocracy}

An autocracy is a system of government in which supreme power is concentrated in the hands of one person \cite{politicalterms}. This becomes fairly straightforward to model by letting the influence allocation be $\alpha(i) = 0$ for any $i$ different to the autocracy dictator, and $\alpha(dictator) = 1$. For the resolution function we just return the event that accumulated 1 of influence, which will be the one that the dictator chose, because it is the only agent with influence more than 0.

\subsection{Democracy}

The term originates from the Greek δημοκρατία (dēmokratía) "rule of the people" \cite{greeklexicon}. This means that every agent will be allocated the same positive non-zero influence value, so for any decision instance using a democratic system: 

$$\alpha(i) = a $$

where $a > 0$. For decision resolution we use the $max_e$ function we used in the last section, which effectively yields the event with most influence accumulated, so

$$\delta(E_{inf}) = max_e(E_{inf})$$

\subsection{Representative Democracy}

Representative democracy is a variety of democracy founded on the principle of elected officials representing the whole community, officials are elected through a democratic collective decision of the community, and then officials decide further decision instances using similar democratic systems between them. This encourages the modeling of decision instance composition and instance dependency.

Lets divide the representative democracy system as a 2-phased system, phase 1) elects officials, and phase 2) resolves further decision instances using elected officials. As one may observe all instances on phase 2 depend on phase 1, because the decision context of phase 2 instances are limited by the resolution of phase 1. In this specific case phase 1 subsets the community, and further instances in phase 2 use such subset instead of the original community. 

More formally, let phase 1 be the decision game $ P_1 = (N, A, u, \alpha, \delta)$, let $C$ be the set of candidates, $A$ some subset of the powerset of candidates $A \subset \mathscr{P}(C)$, and $\alpha$ and $\delta$ the democratic system already defined. Then the democratically elected officials will be $\delta(A_{inf})$, and any decision game $(M, B, v, \alpha', \delta')$ will be a democratic representative decision game of $P_1$ iff $M = \delta(A_{inf})$. 