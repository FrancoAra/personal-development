Decision systems refer to the broad concept of a community of agents that face the necessity of collectively decide over an issue. How can they self organize in order to achieve such task? How can a notion of "justice" be preserved? What is the most efficient method? How can it be automated? All of these are questions that this research line tries to answer.

The most obvious domain for decision systems is government and political systems. New political systems have been proposed like \cite{anewpoliticalsystem}. Also \cite{politicallife} tries to adapt system theory to political science. But both efforts lack mathematical rigor and a true abstract model for decision systems.

Crowdsourced public project planning is another domain example, in \cite{crowdsourcing} it is argued the medium of the Web as an appropriate technology for harnessing widely distributed talent and contributions, but still the lack of a mathematical model impedes further decision system design and development.

The field that fits our research is game theory; the audience and applications for game theory has grown dramatically in recent years, and now spans disciplines as diverse as political sciences, biology, psychology economics, linguistics, sociology and computer science, among others \cite{gametheory}.

A similar subfield to our research is decision theory; decision theory is the analysis of the behavior of an individual (agent) facing nonstrategic uncertainty \cite{decisiontheory}, but the difference with this research line is that we are modeling collective decision making (not individual) in a multi-agent environments, even some times independently of agent utility.

The rest of the draft will continue with the proposal of a formal mathematical model for multi-agent collective decision making, then validating it by modeling well known political systems, also new decision systems are proposed based on the model, and finally future research opportunities are enlisted. 